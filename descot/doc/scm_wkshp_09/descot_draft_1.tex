\input sigplanconf_plain 
\input eplain
\input bnf

\conferenceinfo{2009 Workshop Scheme and Functional Programming}{}
\authorinfo{Aaron W. Hsu}{Indiana University}{awhsu@indiana.edu}
\title{Descot: Decentralizing Scheme Library Repositories}

\maketitle

\beginabstract

Currently, no de facto standard exists for the distribution, 
location, and retrieval of Scheme libraries. 
In fact, the Scheme community 
remains largely unaffected by attempts to unify them
around one library distribution system. Some argue that such 
a system cannot exist given the wide range of implementations 
and the diversity of the Scheme community. This paper documents 
a new approach to library distribution in Scheme that focuses 
on the interface between repositories and tools rather than 
on the tools themselves, tied to a specific repository.  The Descot
system presented specifically avoids restrictions
of implementation or function (such as specific tools). The author 
argues that Descot provides the necessary framework for a successful 
library distribution mechanism to form within 
Scheme's current climate by eliminating the need for a single suite of 
tools for library management, as well as eliminating the need for a 
single code repository. The author discusses some of the benefits 
to this approach, particularly in light of the Scheme community's 
tendency to fracture on the details of 
implementing such a suite. 
Additionally, the current tools written 
for the Descot protocol, the supporting libraries and formats, 
are also discussed.

\endabstract

\section{Overview}

The history of Scheme shows a progression in techniques 
for abstracting over blocks of code from very basic source 
file handling up to today's sophisticated module 
systems. Before module systems were widely used by implementations, 
most Scheme programs had to isolate code into differing files 
and subsequently order their evaluation with
cumbersome constructs such as {\tt eval-when} and {\tt load}. 
Most implementations today have some sort of module system 
that reliably controls the dependencies between libraries, 
making it easier to build large programs.
The R6RS {\tt library} form is the latest module system to 
be widely implemented and standardized.\cite{r6rs}

The R6RS {\tt library} form attempts to provide a portable 
means of sharing code, 
but the requirements of standardization forces the
library form to permit different evaluation models, 
resulting in different semantics for libraries in 
differing implementations\cite{phasing}. 
Furthermore, not all implementations 
of Scheme support the R6RS {\tt library} form. 
Thus, in order to share code, 
it does not suffice to distribute libraries in an arbitrary module form,
because other implementations do not 
understand how to use the library unless they also 
understand the same module system.

In fact, the Scheme community is characterized by a wide 
variety of implementations.
Scheme's expressivity encourages different methodologies, 
but this also strains any attempt 
to gather libraries together for general use by the 
community at large. 
Usually, users are restricted to 
a single implementation's libraries, unless some effort is 
taken to locate, port, and manage libraries outside of the 
implementation's libraries. 
While this can be done, 
it creates a barrier that discourages 
sharing code among Schemers and 
which often discourages 
the use of Scheme in large applications.

Some have attempted to address the need for collaboration by
developing repositories and tools to manage libraries. 
Generally, these applications take
their inspiration from repositories like CPAN\cite{cpan}, 
which have a central repository of code from which users 
may download and install software
packages. These repositories are meant to be tightly integrated
into the workflow of the user, with libraries sometimes being
automagically grabbed and installed whenever they are needed.
Unfortunately, to date, these systems have not succeeded in
establishing any form of standard for library distribution
in Scheme. 

The usual complaints concerning these systems 
center around three principles of library repositories:

\unorderedlist
\li They should contain many up-to-date libraries.
\li The tools should be easy to use.
\li The whole system should integrate well with library authors and users.
\endunorderedlist

Current repositories and their tools fail to adequately 
satisfy these requirements for the entire Scheme community.
Some satisfy the needs of 
only one implementation\cite{planet}, others require excessive changes to existing 
workflows, and still others are hard to use for either authors or 
users.

The value of Scheme's segregated community aside,  most everyone
would like to have a system of library distribution
and integration, even if they disagree on how it should be done.

\section{Necessary Components}

Every library repository requires some minimal set of features and 
tools to be useful. These are:

\unorderedlist
\li A database of libraries
\li A way to publish libraries
\li A way to retrieve and manage libraries for the user
\li A search engine of some kind for locating libraries
\endunorderedlist

Any decent library repository has these features. 
There must be some easy way to submit libraries, 
some easy way to obtain and use them, 
and some way of finding the desired libraries.

The Descot\cite{descot} system acknowledges these components, 
but almost entirely ignores them. 
All current repository systems implicitly provide some interface 
for connecting and communicating between each of these components, 
even though this interface often lays private and hidden behind the 
tools. 
Descot focuses on a public interface of communication between the 
above components, rather than focusing on the above components themselves. 
The justification and motivation behind such an approach will be 
examined after the current approaches are analyzed.

\section{Current Approaches and Attempts}

Current library distribution systems fall into two camps. 
The first attempts to provide implementation neutral tools, 
while the second caters entirely to one implementation alone.

Implementation specific library repositories include PLT's 
PLaneT\cite{planet}, Chicken's Eggs\cite{eggs}, 
and Bigloo's repository\cite{bigloorepo}.  Implementation 
neutral systems include Snow\cite{snow}, CSAN\cite{csan}, 
CxAN\cite{cxan}, and SLIB\cite{slib}.

Implementation specific repositories integrate well,
have good package management tools,
and utilize some extension to Scheme at the language level to 
improve integration.
Implementation independent repositories tend 
not to integrate as well, but provide more portability.
Snow is a well known implementation 
neutral repository.

SLIB illustrates another approach which, 
while seemingly implementation neutral, 
ends up acting more like an implementation in itself.
It does not utilize the native module system, reader, 
macros, or other parts of the Scheme system, and instead 
provides its own set of tools for managing itself. 
This makes it difficult to integrate SLIB libraries into 
existing code, and authors cannot easily share code in a 
portable way with SLIB.

Snow, while being more portable, also enforces a certain toolset, 
framework, and ideology about package management on the user. 
Authors must package their libraries using a Snow specific 
format in order to distribute them through Snow. 
Authors must then maintain two sets of code for their libraries, 
Snow versions, and non-Snow versions.
If everyone used Snow, or if a large enough majority used it, 
the author would not bother to keep a second 
version along with the Snow packaged version, 
but until then, authors must use two packages. 
This makes it difficult for Snow to gather the necessary 
momentum for widespread use because libraries will likely 
become out of date and users still have to use the Snow tools 
provided to access the repository, or decipher the code in 
order to write their own tools.

Each of the systems above suffers in one area or another, 
whether in issues of portability, integration, ease of use, 
developer liberty, or other areas.  
Importantly, developer liberty tends to be an important factor 
when considering the Scheme community as a whole. 
While one set of users may 
be fine with one set of tools on one implementation, such as 
PLaneT, such a system cannot hope to succeed among all of the 
differing ideas and implementations of the entire Scheme 
community at large. 

\section{A Different Approach and Focus}

Why have none of these systems succeeded in garnering the 
support of the majority? 
For some, this was never the goal. 
For others, they fight an uphill battle because of 
presumptions about the way a library system ought to operate. 
All of these systems have a single toolset with a single repository. 
Doing this in CPAN works, because there really is one implementation. 
Scheme is, however, by its nature, a diverse and varied set of 
implementations and users that have very different needs.  These 
needs are not met by trying to enforce a single solution on the 
whole. 

That is not to say that each of the above systems meant to do 
this, but they had little choice when their goals are examined. 
Their goals included such things as tools and an entire 
set of programs necessary to manage libraries, 
and neglect or entirely ignore distribution from the perspective 
of how each of these tools interact with one another. 

Some of these systems already have a nice toolset for use, 
if users are so inclined to use it,
but they do not have the necessary library size 
or flexibility to encourage other users to use their repository.
Descot takes a different approach.
In fact, Descot is nearly orthogonal in focus to systems like 
Snow, and Snow could be ported to use Descot fairly easily.
Descot prioritizes the sharing of information
over the ubiquity of a single toolchain. 
Rather than enforce a single toolset, program, or 
even repository, Descot provides the necessary framework for 
many different tools and repositories to coordinate together. 

This emphasis on the representation and protocol, while 
not sufficient to make a complete system, results in a system 
capable of scaling up to the Scheme community at large. 
Since Descot does not specify a specific server, but a specification 
of how a server should behave,
authors and implementations may run their own repositories, 
and other users and implementations will still be able to understand 
their database.
This grassroots capability also allows for the system to grow 
incrementally, without requiring everyone to start using the 
system right away. In this way, the system has a better chance 
of seeing more people use it, since the burden on the programmer 
is lowered. 

Additionally, Descot does not store or require the storage of 
a library's code, just the metadata. This means that authors 
do not have to have two copies of the code, 
and very little extra work need be done in order to 
publish a library to a Descot repository.

Since Descot uses an extension friendly format 
for recording the metadata about libraries, it allows 
implementation specific extension that can safely be ignored 
by other implementations or users who wish to port the code.
In this way, implementations can 
utilize Descot to provide access to libraries, while at the 
same time providing a much higher degree of integration with 
elements that may only suit that implementation, 
without forcing the rest of the world to support them. 

In summary, Descot provides a specification of interaction 
that is tool and implementation independent. 
While some tools are already written for this specification, 
their existence and use is neither required nor even 
that beneficial if other tools are better suited. 
However, new tools and management suites written using 
Descot's specification of behavior can be trusted to operate 
correctly with existing tools and repositories. 

This approach caters to the independent nature of the Scheme 
community and permits a variety of solutions to library management 
while guaranteeing the ability for each of these solutions to work 
together with others. Put another way, Descot is an attempt to 
``document the chaos'' of libraries in the Scheme community, 
rather than trying to reign in the chaos. 

\section{Benefits and Reasons to Use Descot}

It may seem that Descot's benefits are purely abstract 
or theoretical. 
Since Descot doesn't try to solve the more commonly 
addressed issues of library management suites and tools, 
Descot's benefits might appear marginal.
Apparently, no single library management 
suite, repository, and all-in-one approach will gain 
the traction necessary to nontrivially affect the 
community's way of operating. 
Descot's advantage comes in its ability to leverage all 
the existing tools, as well as the labor of the entire 
community, rather than relying on a few select 
individuals to create a critical mass. 
It enables communicating library information to the 
community in a human and machine readable 
fashion, while letting users and authors do things the 
way they want.  Descot provides the glue, nuts, bolts, fittings, and 
standards, as it were, to allow each party of the Scheme 
community to contribute in a compatible, integrated way 
to a library system, so that the Scheme community as a 
whole can benefit. 

The benefits Descot provides correlate to the three 
complaints most people have about existing solutions.

Firstly, it encourages up-to-date, direct-from-the-source 
libraries, and makes it easy for authors to publish 
their libraries in a machine and human readable way, 
confident that they will be found, indexed, and located 
through the web of Descot Servers on the Internet. 
This solution is more effective than using an existing 
web search engine because the searching is very specific 
to Scheme Libraries. Consequently, more tools will be 
able to access more libraries than was previously possible. 
This benefits both authors and users, since it improves the 
overall availability of libraries to the community. 

Since Descot does not require submission to a central 
repository, prolific authors can host their own repositories, 
and have them neatly integrated into their workflow, 
making the changes available that much faster to the 
community as a whole, and reducing the amount of work for 
the author in order to ensure that each mirroring
repository has a copy of the latest library.

Secondly, Descot is easier to use because the tools 
that can be written can focus directly on handling the 
needs of a specific community, rather than trying to 
handle the needs of an entire programming community. 
This means easier to use, simpler tools for the 
user, with more availability. Likewise, authors can 
use the specific tools they like to manage their own 
code, and know that their tools will interoperate with 
all the other Descot-compatible tools and repositories. 

Not only does the possibility of many specific tools 
benefit the system's ease of use, but it also means that 
there is a much higher chance that an user can find or 
make a tool to integrate into his or her existing workflow
without much difficulty. 

Since all these libraries can now be made easily visible, 
and can be easily managed, this will encourage more 
people to use Scheme and contribute libraries who may have 
previously avoided using Scheme because of the difficulty 
of finding and managing libraries. 

In short, Descot addresses the issues that have made it difficult 
for existing repositories to gain traction in the Scheme community, 
enabling better cooperation between systems, and making it easier 
to see what libraries exist. 

\section{Library Description Language}

The actual library description language has been written using RDF\cite{rdf}. 
RDF is a framework for documenting the metadata about some resource. 
In this case, it is used to document libraries and the subcomponents 
thereof. 

RDF is at its core a directed graph where a node 
pointing to another node forms a sentence, with the subject being 
the originating node, the object being the target node, and 
the arrow representing the predicate. One can then assign 
meaning to the predicate, subjects, and objects. Each node and 
edge is identified by either a blank node or a globally unique 
URI. \cite{rdfprimer} 

Descot describes a number of classes and properties under a given 
URI prefix which have assigned semantic meaning in the Descot 
system. Nodes which belong to certain classes are expected to 
have certain properties associated with them, and the 
set of properties defined by the Descot Schema \cite{descotschema} 
form the predicates 
that are needed above and beyond the basic RDF properties \cite{rdfschema}
to describe library information. 

This section uses the Turtle format \cite{turtle} for describing RDF Graphs, 
but the next section will detail a new S-expression based format 
that is expected to be more appealing to Schemers. In fact, Descot 
is independent of a particular format for representing the 
Descot RDF Graphs so long as the two servers understand the format 
being used to communicate said graph. Current Descot APIs support 
any format for RDF Graphs storage.

Descot currently defines classes for Repositories, Libraries, 
Bindings, Licenses, People, Retrieval methods (such as 
Archive, CVS, and Single File), and Implementations. It also 
defines properties such as a node name, alternatives, 
description, homepage, dependencies, exports, license, 
copyright information, authors, modification dates, maintainers, 
version, and location information. These form a set from which 
libraries are described. 

A few properties have codomains that take other nodes such as People, 
bindings, and locations.  These are particularly important, because
things like binding nodes allows an arbitrarily large amount of 
information to be associated with a binding, and not just its 
name.  This means that potentially, two libraries may match the 
same search for a procedure, even if they export the procedure 
with different names. Likewise, Locations are extensible, so 
that implementations and servers can provide more information 
about how to obtain and/or install a library than is by default 
provided with the Descot schema. 

Here is a simple example of a library, where the 
following prefixes are assumed to be defined: {\tt authors:}, 
{\tt impls:}, {\tt licenses:}, {\tt dscts:}, 
{\tt libs},
{\tt rdf:}, and {\tt xsd:}. It is important to note that 
prefix names are arbitrary and do not matter so long as they 
expand into the correct URIs. 

\medskip
\verbatim
libs:system/malloc#chez
rdf:type dscts:Library ;
dscts:name "Garbage Collected Malloc" ;
dscts:names ( "malloc" "gc-malloc" ) ;
dscts:contact authors:arcfide ;
dscts:categories ( "system" ) ;
dscts:desc 
  "Allocate garbage collected memory." ;
dscts:version "1.0" ;
dscts:authors ( authors:dybvig );
dscts:copyright-year "2008"^^xsd:gYear ;
dscts:copyright-owner authors:dybvig ;
dscts:license licenses:public-domain ;
dscts:implementation impls:chez ;
dscts:modified 
  "2009/03/08 23:39:18"^^xsd:dateTime .
dscts:exports (bindings:gc-malloc);
dscts:deps (libs:system/ffi#chez);
dscts:location _:location.

_:location rdf:type dscts:CVS ;
dscts:cvs-root 
  "anoncvs@anoncvs.sacrideo.us:/cvs";
dscts:cvs-module "lib/malloc.ss" .
|endverbatim

The above defines a single library that exports a garbage 
collected {\tt malloc} procedure. Notice also that the download 
location (a blank node) is a CVS node pointing to some 
resource in CVS. 

\section{S-expression RDF}

SRDF is an s-expression based format for describing RDF Graphs.  It is
meant to be mostly equivalent in its form to Turtle. Since the language
is S-expression based, it is easier for Scheme and Lisp parsers to
parse it. Parsers for other languages can also be written very
easily. This makes it particularly nice for use in automated systems
or in areas where S-expressions are the natural representation format.
SRDF is designed to work for most Scheme's {\tt read} procedure. 

SRDF documents are composed of a series of RDF triples and possibly,
prefix definitions. Prefixes take the form
{\tt (= name "uri")}, and associate a given Scheme
symbol with a URI string. Otherwise, the form is an RDF triple or a set
of triples.

Normal triples are just a list of three elements, each a URI. However,
it is possible to have a subject associated with more predicate
and objects by replacing the list that would hold the single
predicate and object with a list of such predicates and objects.
Likewise, one can specify more objects to be associated with a
given subject and predicate by doing the same thing with the
object list, and replacing the {\tt cdr} that would normally hold the
object with a list of such objects.

If the {\tt cadr} of a predicate pair contains a list of objects, this
represents a collection of objects, and is created in the same
way that a turtle collection syntax is created: by associating
a series of blank nodes with the right predicates with each of
the objects listed.

It is important to note the difference between an object that
is a collection, and a list of objects that are each associated
with the subject and predicate. The following is an instance of
the former:

\medskip\verbatim
("subject-uri" "predicate-uri"
  ("object1" "object2" ...))
|endverbatim
\medskip

\noindent Whereas the following is an instance of the latter:

\medskip\verbatim
("subject-uri" "predicate-uri"
  "object1"
  "object2"
  "object3")
|endverbatim
\medskip

Normal RDF triples take the form:

\medskip\verbatim
("subject-uri" "predicate-uri" "object-uri")
|endverbatim
\medskip

A blank node may be inlined into the graph by using a `*' as the
beginning symbol in an object context like so:

\medskip\verbatim
("subject" "pred" (* "pred" "object"))
|endverbatim
\medskip

Of course, blank nodes may have anything that is a valid predicate
{\tt cdr} as its {\tt cdr} so the following is also valid:

\medskip\verbatim
("subject" "pred" 
 (* ("pred1" "object1") ("pred2" "object2")))
|endverbatim
\medskip

URIs may be described by their full path names as strings,
as prefix combined paths, or as blank node paths. The following
are all valid URIs:

\medskip\verbatim
"http://some.domain/path/to#blah"
"blah"
(: prefix "blah")
(_ "uniqueid")
|endverbatim
\medskip

We use `:' for prefixes and `\_' for blank nodes.

In addition to URIs, we permit literals as valid {\tt car}s for objects.
Literals can be strings, numbers, booleans, or may be strings
with either languages or types associated with them. The following
are examples of languages and types, respectively:

\medskip\verbatim
($ "Language unspecified.")
(& "English Sentence lies here." en)
(^ "2008/01/03 14:00" (: xsd "date"))
|endverbatim
\medskip

The following is a fairly formal BNF grammar with the exception of
tokens such as strings, numbers, and booleans being undefined and
presumed to be defined lexical values. Additionally, we define
S-expression in terms of atoms and pairs, so the BNF grammar is
also defined in the ``longhand'' notation for pairs and lists.
This means that while the BNF Grammar states something
like {\tt ("subj" . ("pred" . ("obj" . ())))} as the
valid simplistic RDF triple, it is also legal in practice to use
the shorthand version of this: {\tt ("subj" "pred" "obj")}

\begingroup\narrower
\begingrammar
%
<rdf sexp>:     <rdf triple> | <rdf triple> <rdf sexp>.\par

<rdf triple>:   "(" <uri> "." <rdf subject tail> ")" ; | "(" "=" name string ")".

<rdf subject tail: <rdf predicate> ; | <rdf predicate list> .

<rdf pred list>: "()" ; | "(" <rdf predicate> "." <rdf pred list> ")".

<rdf predicate>: "(" <uri> "." <rdf object list> ")".

<rdf object list>: "()" ; | "(" <rdf object> "." <rdf object list> ")".

<rdf object>: <uri> | <literal>
        ; | <rdf object list> | <blank node list> .

<blank node list>: "(" "*" "." <rdf subject tail> ")".

<uri>: uri ; | "(" ":" "." <uri list> ")" ; | "(" "\_" name ")" .

<uri list>: "()" ; | "(" <uri name> "." <uri list> ")".

<uri name>: uri | name.

<literal>: number | boolean ; | "(" "\$" string ")" ; |
        "(" "\&" string name ")" ; | "(" "\^{}" string <uri> ")".

\endgrammar
\endgroup
\endgroup
\medskip

The previous example given in Turtle format, could be rewritten as 
an SRDF Document as follows:

\medskip
\verbatim
((: libs "system/malloc#chez")
 ((: rdf type) 
  (: dscts "Library"))
 ((: dscts "name") 
  ($ "Garbage Collected Malloc"))
 ((: dscts "names") 
  (($ "malloc") ($ "gc-malloc")))
 ((: dscts "contact") 
  (: authors "arcfide"))
 ((: dscts "categories") 
  (($ "system")))
 ((: dscts "desc") 
  ($ "Allocate garbage collected memory."))
 ((: dscts "version") 
  ($ "1.0"))
 ((: dscts "authors") 
  (: authors "dybvig"))
 ((: dscts "copyright-year") 
  (^ "2008" (: xsd "gYear")))
 ((: dscts "copyright-owner") 
  (: authors "dybvig"))
 ((: dscts "license") 
  (: licenses "public-domain"))
 ((: dscts "implementation") (: impls "chez"))
 ((: dscts "modified") 
  (^ "2009/03/08 23:39:18" 
     (: xsd "dateTime")))
 ((: dscts "exports") 
  ((: bindings "gc-malloc")))
 ((: dscts "deps") 
  ((: libs "system/ffi#chez")))
 ((: dscts "location") 
  (* ((: rdf "type") (: dscts "CVS"))
     ((: dscts "cvs-root")
      ($ "anoncvs@anoncvs.sacrideo.us:/cvs"))
     ((: dscts "cvs-module") 
      ($ "lib/malloc.ss")))))
|endverbatim

\section{Server Interface}

The server interface for Descot is actually quite simple. A server 
hosting a repository should provide a single URI which, when accessed, 
will provide a repository consisting of a set of library nodes 
whose only predicates are modified times.  This can be used to 
mirror the library repository. Additionally, any Descot resource 
URI, such as a library, author, binding, or location node, when 
accessed (such as via HTTP) will return all of the properties that 
have that URI as their subjects.  In this way, it is possible to 
walk through from one update URI to all the rest of the repository, 
and obtain all the relevant information. 

Additionally, when returning information about a URI, the graph must 
be recursively processed for blank nodes connected to the subject 
URI, since these nodes will not be accessible outside of the current 
context. Their graphs must also be returned in some format along 
with the originating URI.
Another way of seing this is that, given a URI, 
when a Descot server fulfills a request for this URI, 
it must provide all the nodes spawning from the URI 
and their children and so forth up to the point where the child node 
is itself a URI which is accessible independently.

For example, in the above section, a library is defined with an 
author node, but this author node has its own unique URI that is 
accessible globally, and so a Descot server which handles request 
for the above library should only print out that node, and not 
its children, since a separate request can be made for the author 
node's children. However, the location predicate of the library 
points to a blank node with a number of children, and these children 
would have to be provided, since there is no other way to access 
them, because blank nodes are not globally accessible. 

\section{Query Servers}

An additional, suggested but not required feature for Descot is 
the query server. This is a SPARQL \cite{sparql} engine that will 
accept queries using the SPARQL language.  
In this way, clients can provide search 
over repositories without necessarily needing to cache the entire 
database.  It can also serve as an useful backend for user 
queries on such clients as web applications. 

\section{Client-side tools}

Additional tools are necessary to make Descot the useful system 
that it can be. These would include CLI package management tools, 
of which Snow comes immediately to mind. Snow could easily be 
extended to use the Descot format and to communicate with 
Descot servers. This would allow users to leverage the existing 
code base in Snow, while gaining from the new approach Descot 
takes. 

A web interface has already been designed built on top of mod\_lisp 
\cite{modlisp}
and Chez Scheme \cite{chez} which provides an interface to the current 
Descot repository through a web browser.  It is still in its 
infancy, but is functional for searching and browsing. 

The author hopes that PLaneT, Eggs, and other repositories 
will decide to provide access to their content through the Descot 
server protocol, to allow these libraries to be mirrored 
to other services, and open up the possibility of future 
portability. 

Because Descot's language can be used just about anywhere, 
the authors would like to see crawlers for Usenet and 
other services that would pick up Descot content. 

Currently, it is not obvious how to set up
one's own Descot server.  However, when Descot's implementation 
reaches maturity, additional tools to ease deployment will 
be made available.  These will make it easy to run Descot 
from just about any web server withoug having to have new tools 
running or making modifications to the web server.

\section{Extending and Integrating Descot}

It would be nice to see Descot used as an embedded metalanguage 
inside of Scheme code, so that distributions of Scheme code 
could gain enough information about them to automatically put 
them into Descot without further user intervention. Since Descot 
is built on RDF technologies, extending the language to handle 
new requirements does not take much effort, and can be done 
without harming existing tools. The emphasis on 
the format rather than the tool allows Descot to be integrated 
into a wide variety of tools and applications, facilitating 
better communication about library metadata in many areas.

\section{Current Status}

Currently, the core libraries for Descot are complete, and are 
publically available via CVS. An API with documentation exists 
to facilitate the creation of other Descot Servers, and libraries 
for handling Descot Stores are available. 

A web client and server utilizing these libraries is also available
and running. It accepts both submissions to its repository and
provides information about the current libraries in its store. These
are slowly being populated.

The code for the system is only available via CVS at the moment, 
but the Descot libraries will be released once the system has 
stabilized to an acceptable point. Users are encouraged to 
submit comments and suggestions, as Descot is meant to 
satisfy the needs of the Scheme community at large, and not one 
specific group over another. 

\section{Future Work}

Once the web application is stable, a SPARQL query engine is 
planned, which will allow more sophisticated queries from clients. 
A CLI client is planned, probably a modification of Snow, if 
possible. Writing Descot interfaces to existing repositories, such 
as Chicken's Eggs and PLT's PLaneT are also desirable. 

After the Descot system stabilizes to a mostly useful state, 
the author intends to pursue the issue of taking the library 
meta data and extending it enough to assist in the automated 
or semi-automated porting of libraries from one implementation 
to another. This, however, is an issue that needs much more 
work, and requires a well accepted Descot system before it 
can be used successfully.

\section{References}

\bibliography{descot}
\bibliographystyle{plain}

\bye
